% This must be in the first 5 lines to tell arXiv to use pdfLaTeX, which is strongly recommended.
\pdfoutput=1
% In particular, the hyperref package requires pdfLaTeX in order to break URLs across lines.

\documentclass[11pt]{article}

% Remove the "review" option to generate the final version.
\usepackage[review]{emnlp2021}

% Standard package includes
\usepackage{times}
\usepackage{latexsym}

% For proper rendering and hyphenation of words containing Latin characters (including in bib files)
\usepackage[T1]{fontenc}
% For Vietnamese characters
% \usepackage[T5]{fontenc}
% See https://www.latex-project.org/help/documentation/encguide.pdf for other character sets

% This assumes your files are encoded as UTF8
\usepackage[utf8]{inputenc}

% This is not strictly necessary, and may be commented out,
% but it will improve the layout of the manuscript,
% and will typically save some space.
\usepackage{microtype}

% If the title and author information does not fit in the area allocated, uncomment the following
%
%\setlength\titlebox{<dim>}
%
% and set <dim> to something 5cm or larger.

\title{Instructions for EMNLP 2021 Proceedings}

% Author information can be set in various styles:
% For several authors from the same institution:
% \author{Author 1 \and ... \and Author n \\
%         Address line \\ ... \\ Address line}
% if the names do not fit well on one line use
%         Author 1 \\ {\bf Author 2} \\ ... \\ {\bf Author n} \\
% For authors from different institutions:
% \author{Author 1 \\ Address line \\  ... \\ Address line
%         \And  ... \And
%         Author n \\ Address line \\ ... \\ Address line}
% To start a seperate ``row'' of authors use \AND, as in
% \author{Author 1 \\ Address line \\  ... \\ Address line
%         \AND
%         Author 2 \\ Address line \\ ... \\ Address line \And
%         Author 3 \\ Address line \\ ... \\ Address line}

\author{First Author \\
  Affiliation / Address line 1 \\
  Affiliation / Address line 2 \\
  Affiliation / Address line 3 \\
  \texttt{email@domain} \\\And
  Second Author \\
  Affiliation / Address line 1 \\
  Affiliation / Address line 2 \\
  Affiliation / Address line 3 \\
  \texttt{email@domain} \\}

\begin{document}
\maketitle
\begin{abstract}

\end{abstract}

\section{Introduction}
\subsection{open questions}
\begin{itemize}
    \item Can we still call it consistency? We could report conditional violation only based on the rules it captures. 
    \item Can we motivate this beyond probing?
\end{itemize}

\subsection{why?}
\begin{itemize}
    \item better accuracy
    \item trust
    \item more intelligent
\end{itemize}

\subsection{difference to prior work}
\begin{itemize}
    \item global consistency
    \item F1 improvements
    \item continuous resolution 
    \item identify regions where more rules are needed
\end{itemize}

\subsection{Future work}
\begin{itemize}
    \item extract rules from text or the model itself
    \item 
\end{itemize}

\subsection{contributions}
\begin{itemize}
    \item rule dataset
    \item mutual exclusive rules are not well captured by the model
    \item ConcepNet rules are captured but not consistently applied
    \item common sense knowledge base identification from a noisy PLM/ mechanism to compose rules and assertion to improve F1 and consistency
\end{itemize}
\section{Data}
Rules are generate from ConcepNet \cite{Speer2017ConceptNet5A}.
Relations: IsA, CapableOf, HasParts, HasProperty, MadeOf, HasA.

Two types of rules: transitive rules and mutual exclusive rules.

Subjects from WordNet \cite{wordnet}.
\section{Calibration}
\section{Results}
\subsection{Consistency}
\begin{table*}
\begin{tabular}{ l|c|c} 
& consistency model & consistency resolved \\\hline
transitive rules & & \\\hline
mutual exclusive rules & & \\\hline
both & & \\\hline
\end{tabular}
\caption{Consistency before and after wSAT solving}
\end{table*}
\subsection{Accuracy}
\begin{table*}
\begin{tabular}{ l|c|c} 
& F1 model & F1 resolved \\\hline
gold facts & \\\hline
deducible facts& \\\hline
non-deducible facts&& \\\hline

\end{tabular}
\caption{F1 before and after wSAT solving model rules.}
\end{table*}
\subsection{Model rules vs. gold rules}
\subsection{Incremental}
\subsection{Self diagnosis}
Identifying regions where more rules are needed or where model rules need human intervention.
\section{Related Work}
\section*{Acknowledgements}


% Entries for the entire Anthology, followed by custom entries
\bibliography{anthology,custom}
\bibliographystyle{acl_natbib}

\end{document}
